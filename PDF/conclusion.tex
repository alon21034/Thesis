\begin{CJK}{Bg5}{bsmi}

%---------------------------------------------
%	Chapter Conclusion
%---------------------------------------------

\chapter{Compare to Other Schemes}

This section is going to compare this new scheme to other related schemes introduced in section~\ref{sec:related-work}. Due to the threat model is different in different scheme, we can only select some general criteria to do a brief analysis. 

For the estimation in usability, it takes the traditional password-based scheme as a standard. When the estimation in one criteron of a scheme is better than password-based scheme, it will be marked as a '+' sign, or be marked as a '-' on the contrary. If it is as good as the password-based scheme, it will be marked as a '=' sign. 

For the comparison of deployment ability, the scheme will get a 'O' sign if it has the corresponding property, or get a 'X' sign if not.

\begin{table}[h]
\begin{tabular}{|c|c|c|c|c|c|c|}
\hline
                           & pwd & my scheme & SecurID & YubiKey & 2-step & Smard-card \\ \hline
physical observation       & X   & O         & O       & O       & O       & O          \\ \hline
targeted impersonation     & X   & O         & O       & O       & O       & O          \\ \hline
guessing attack            & X   & O         & O       & O       & O       & O          \\ \hline
internal observation       & X   & O         & O       & O       & X       & O          \\ \hline
leaks from other verifiers & X   & O         & O       & O       & O       & O          \\ \hline
phishing                   & X   & O         & O       & O       & O       & O          \\ \hline
\end{tabular}
\caption{Comparison of Seurity}
\end{table}

\begin{table}[h]
\begin{tabular}{|c|c|c|c|c|c|c|}
\hline
                   & pwd & my scheme & SecurID & YubiKey & Google 2-step & Smard-card \\ \hline
memorywise-effort  & =   & +         & +       & +       & =             & +          \\ \hline
scalable ability   & =   & +         & -       & =       & =             & -          \\ \hline
nothing to carry   & =   & -         & -       & -       & -             & -          \\ \hline
Physical-effort    & =   & =         & =       & =       & =             & =          \\ \hline
Efficiency         & =   & =         & =       & =       & -             & =          \\ \hline
Correctness        & =   & =         & =       & =       & =             & =          \\ \hline
recovery-from-loss & =   & -         & -       & -       & =             & -          \\ \hline
\end{tabular}
\caption{Comparison of Usability}
\end{table}

\begin{table}[h]
\begin{tabular}{|c|c|c|c|c|c|c|}
\hline
                   & pwd & my scheme & SecurID & YubiKey & Google 2-step & Smard-card \\ \hline
Accessible         & O   & X         & X       & O       & X             & X          \\ \hline
User-cost          & O   & X         & X       & X       & X             & X          \\ \hline
Server-compatible  & O   & O         & X       & X       & X             & X          \\ \hline
Browser-compatible & O   & O         & O       & O       & X             & O          \\ \hline
\end{tabular}
\caption{Comparison of Deployment Ability}
\end{table}

\chapter{Conclusion}

In this study, it proposes a user-friendly authentication solution. With the analysis and the comparison result in previous chapters, this new scheme can resist to lots of general attack methods. In chapter~\ref{cha:discussion}, it also claimed that this scheme is secure enough under a reasonable threat model. 

For the usability, the only disadvantage of this scheme is as same as others device-based or smart-card-based schemes: users need to bring a token or device with them, and it is hard to recover if the device got lost. In consideration of the recovery-from-loss property, there are some approaches can solve this problem. For example, users can register their e-mail address to verification server, and use OTP through e-mail to reset their UUID and private key.

For the deployment ability, though it didn't perform well in both accessibility and user-cost, server-compatible and browser-compatible are much more important. Thus, we can still say that this scheme has a good deployment ability.

\end{CJK}