%\documentclass[12pt]{article}
%\usepackage{CJK}
%\usepackage{pinyin}

\begin{CJK}{Bg5}{bsmi}

%---------------------------------------------
%	Chapter Introduction
%---------------------------------------------


\chapter{Introduction}

In the age of information, more and more services are provided through the Internet. For these web services, it is important that how to verify users' identity on the web. Therefore, a secure \emph{Authentication System} is essential for each service. Nowadays, the password-based scheme is the most common authentication system. However, it is not secure enough\cite{password-security}. Malicious attackers can get people's password in many different approaches. For example, in 2014 May, news said that the well-known website eBay is hacked and more than 200 millions users are affected\cite{ebay-hack}. Since password-based scheme cannot protect us very well, our privacy are exposed to great danger.

\section{Motivation}

Lots of researchers have demonstrated various authentication methods in recent decades, but some of them are not secure enough. Malicious attackers can get people's password in many different approaches. Moreover, attackers can even steal the private information of users, for example, by key-logging or phishing attack. In brief, several systems have severe security issue. In addition to the security issue, some system has lower usability, even if the original password-based scheme\cite{password-usability}. It means that it is not user-friendly. For instance, some graphic-based scheme like PassGo\cite{passgo}. In conclusion, these systems are hard to use.

Furthermore, in some situation, we have to use a public PC in office or library and cause the threating increase dramatically. According to these reasons mentioned above, the aim of the thesis is to develop an authentication system with high security and usability. I use Digital Signature Algorithm (DSA) to prevent the security issue. Then I take advantage of mobile device to build a user-friendly system.

Compared to existing system, our authentication system is much secure and usable. The rest of the thesis is organized as follows. In chapter 2, I introduce some related preliminaries. In chapter 3, I will describe my system more detailed. In chapter 4, I build some criteria to estimate the performance of my authentication system in security, deployment ability, and usability. Then compare it to the password-based scheme and show that there are some improvements by my new design. In the last chapter, I compare some existing token-based authentication schemes to my scheme, and claim that the proposed system is useful enough by comparison.

\end{CJK}
%%% Local Variables: 
%%% mode: latex
%%% TeX-master: "paper"
%%% End: 
