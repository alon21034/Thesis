%\documentclass[12pt]{article}
%\usepackage{CJK}
%\usepackage{pinyin}

\begin{CJK}{Bg5}{bsmi}

%---------------------------------------------
%	Chapter Introduction
%---------------------------------------------


\chapter{Introduction}

In the age of information, more and more services are provided through the Internet. For these web services, it is important that how to verify users' identity on the web. Therefore, a secure \emph{Authentication System} is essential for each service. Nowadays, the password-based scheme is the most common authenication system. However, it is not secure enough, Malicious attackers can get people's password in many different approaches. [TODO example of attack] Since password-based scheme cannot protect us very well, our privacy are exposed to great danger.

\section{Motivation}

Lots of researchers have demonstrate various authenication methods in recent decades. However, some of them are not secure enough, and some of them are with low usability. It is hard to design a solution which is both user-friendly and secure enough.

Therefore, in my research, I try to take the advantage of the mobile device and design a user-friendly authentication solution. 

Lots of researchers have demonstrate various authenication methods in recent decades, but some of them are not secure enough. Malicious attackers can get people's password in many different approaches.Moreover, attackers can even steal the private information of users. For example, [TODO example]. Besides, [TODO example] In brief, there are several security weaknesses in the current systems. In brief, several systems have severe security issue.

In addition to the security issue, some system has lower usability. It means that it is not user-frienly. For instance, [TODO example]. In conclusion, these system is hard to use.

According to these reasons mensioned above, the aim of the thesis is to develop a authentication system with high security and usability. I use Digital Signature Algorith to prevent the security issue. Then I take advantage of mobile device to build a user-friendly system.

Compared to existed system, our authentication system is much secure and usable. The rest of the thesis is organized as follows. In chapter 3, I will describe my system more detailed. In chapter 4, I build some criteria to estimate the performance of my authentication system in security, deployment ability, and usability. Then compare it to the password-based scheme. In the last chapter, I compare some existing token-based authentication schemes to my new design, and the proposed system is much useful by comparison.

\end{CJK}
%%% Local Variables: 
%%% mode: latex
%%% TeX-master: "paper"
%%% End: 
