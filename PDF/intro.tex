%
% $Id: intro.tex 2011-05-03 21:42:00Z by KBJ$
%
%\documentclass[12pt]{article}
%\usepackage{CJK}
%\usepackage{pinyin}

\begin{CJK}{Bg5}{bsmi}
\chapter{Introduction}

Various services are provided through internet.

Some problems in these year:
	privacy
	identification

password-based scheme is the most common solution, but it is not secure enough and is not user-friendly



Researchers demostrate various of solutions, some of them are not user-friendly.

Because of the popularity, mobile devices can provide high convience.

\begin{comment}
In the age of information, more and more services are provided through the internet. Thought it brings us a convient daily life, there are some problems emerge. First, the convience of internet make the bondaries between people blur. Therefore, it is an important issue about how to protect our privacy.
The second problem is that
\end{comment}

\section{Motivation}

In these decades, many researchers demonstrate various methods and try to provide a better authentication scheme. Because it is an internet security problem, some solutions put too much emphasis on security so ignore the convience and usability. Therefore, by the high usability of mobile devices, I try to design a user-friendly authentication scheme.


\begin{comment}
As authentication system is an important part in the world of internet. A good authentication system may protect everyone's privacy not be invaded by the malicious person. 
So far, because of the easy design, the password-based scheme is the most common solution about authentication issue. However, there were some researches demontrates that this was not a proper solution in both security and usability. 
\end{comment}


\section{Contributions}

\chapter{Preliminaries}

\section{Digital Signature Algorithm}

\section{Android HCE Feature}

Many Android devices which offer NFC functionality also support card emulation feature. In most cases, this feature is achieved by a seperate chip, called \emph{secure element}. The Android system is only provide an interface. Therefore, no Android application can involved in the transaction between secure element and the reader. After the transaction complete, an applications can query the secure element directly to get transaction status and notify users. This is why Tte original NFC card emulaiton functionality also called hardware card emulation.

Because this mechanism needs an extra hardware in devices, most of Android application developers cannot take advantages of card emulation feature. To solve this, Android 4.4 provides an additional method of card emulation, which is not invoved with secure element, called \emph{host-base card emulation}. This method allows Android application can communicate with NFC reader directly, not 

\section{Existing Solutions}

Since researchers have studied authentication system for years, there are various of solution now. 

\subsection{Token-based scheme}

\subsubsection{SecurID}

SecurID, now known as RSA SecurID, is a mechanism developed by RSA (the Security Division of EMC) for performing two-factor authentication for a user to a network resource. The following paragragh will describe how it works simply.

Each device stored a defferent secret \emph{seed}, and the back-end server also know this seed. In every 60 seconds, SecurID will generate an 6-digit authentication code according to its seed, and display on the screen. If a user authenticating to a network service, he have to enter both a personal identificaiton number and the 6-digit \emph{code at that moment}. The server, which also has a real-time clock and a database of valid tokens with the associated seed records, authenticates a user by computing what number the token is supposed to be showing at that moment in time and checking this against what the user entered.

However, in March 2011 attackers compromised RSA’s back-end database of seeds, which allowed them to predict the authentication codes generated by any token at any time. This attack forces RSA Security to replace almost every one of the 40 million SecurID tokens in use.

\subsubsection{YubiKey}

\subsection{Mobile-device-based scheme}

\subsubsection{Google 2-step}

\subsection{Smart-card-based scheme}

\subsubsection{Swarn's Smart-card-based Scheme}

\chapter{Architecture of This System}

\section{User Flow}

\subsubsection{Register Phase}

	1. 	User start the initialzization process on his mobile device
		i.  set PIN code
		ii. generate key pair
	2.	User send a registration request to server from browser

	3.	User send his id and public key (and other required credentials) to server.

\subsubsection{Login Phase}

	1. User send a login requesr to server

	2. Server return a nonce ({server info || randombits}) back to browser

	3. The brower pass this nonce to reader application
		i.	The reader start to scan NFC cards.
		ii.	Timeout: 30 seconds

	4. Mobile device ask user to input the correct PIN code and confirm the server information
		i.	Enable card emulation mode right after receive correct PIN code

	5. Mobile device get nonce from reader application
		i.	mobile device signed the nonce with correspond private key.
		ii.	mobile device return the signed-nonce back to reader application

	6. Reader application return signed-nonce back to server.

\subsubsection{Verification Phase}

	1. Server find the corresponding public key accordding to id.

	2. Verify the signature.

	3. Return the result.

\section{Implementation}

\chapter{Discussion}

\section{Security Analysis}

\begin{comment}
This is the most important part of an authentication system.
We have to define our threat model before we start to analyze.
There are 4 components in the scheme I proposed.
\end{comment}

\section{Usability Analysis}

\begin{comment}
The usability can not be neglected when researchers trying to design a system.
Usability is a subjective perception, it may be different from person to person.
The following paragragh states the criterias I used to estimate the usability of a system.
\end{comment}

\section{Deployment Ability Analysis}

\begin{comment}
The deployment ability is also an important thing which is need to be considered, especially in designing an authentication system.
A system with high usability means it is friendly to users; a system with high deployment ability means it is friendly to the system provider or, more precise, the developers.
The following paragraph states the criterias I used to estimate a system's deployment ability. 
\end{comment}

\chapter{Conclusion}

\section{Compare to Other Schemes}

\end{CJK}
%%% Local Variables: 
%%% mode: latex
%%% TeX-master: "paper"
%%% End: 
