\begin{CJK}{Bg5}{bsmi}

%---------------------------------------------
%	Chapter System Architecture
%---------------------------------------------

\chapter{System Architecture}

In this chapter, I am going to explain my design more detailed. My scheme is based on DSA to provide the security and I'll explain it in the first section. The second section exhibits the user flow in three phase: \emph{register}, \emph{login} and \emph{verification}. The last section presents two demostrations about how to apply this scheme. One is for building a website which support this authentication scheme, the other is for how does this scheme cooperating with the existed website.

\section{System Overview}

Let us recall the autehentication process about password-based scheme. As the fig~\ref{fig:password-based-flow} shows, the client give his username and password to server (password is encrypted), and the server checks whether it is valid according to its database. Fig~\ref{fig:my-scheme-flow} is the authentication process of my scheme. Because the verificaiton server should be a passive element, client should send a login request first. After receive a login request, the server send a random nonce back to client. Client generate a signature for this nonce and return to server. Server, then, use the public key to check whether the client is valid or not.

\begin{figure}
\centering
\subfigure[password-based scheme]{
\label{fig:password-based-flow}
\includegraphics[scale=0.6]{picture/password-based-flow.png}
}
\subfigure[my scheme]{
\label{fig:my-scheme-flow}
\includegraphics[scale=0.6]{picture/basic-idea-flow.png}
}
\caption{Authentication flow}
\end{figure}

This is how I used Digital Signature Algorithms in an authentication process. The advantages is that the data communicated between client and server do not need to be encrypted. The only \emph{secret} is private key, which is stored in client's storage. The disavantage of my scheme is that users will need a device to help them creating signature and manage their public keys. Therefore, the use of mobile device is the core of this scheme, bring us a high usability. The user flow become fig~\ref{final-flow} with the help of mobile device.

\begin{figure}
\centering
\label{fig:final-flow}
\includegraphics[scale=0.65]{picture/final-flow.png}
\caption{Authentication flow with mobile device}
\end{figure}

\section{User Flow}

In this section, I'll seperate the autentication process into three parts: \emph{register phase}, \emph{login phase} and \emph{verification phase}.

\subsubsection{Register Phase}

\begin{enumerate}
\item Start the initialzization process on his mobile device, that is, set PIN code and generate key pair.
\item User send a registration request to the verification server.
\item Server return the server information to user and pass it to mobile device via reader applicaiton.
\item Mobile device saved the server information and the private key together, and return the device UUID and corresponding public key back to user.
\item User send the id and public key (and other required credentials required by server) to server.
\item Server saved UUID and public key into its database.
\end{enumerate}

In step one, users need to set a PIN code in order to resist to theft. The key pair is generated by RSA in my implementation, but it can be replaced by any asymmetric key encryption algorithm. In step three, the reason why the server has to add server info into the nonce is to resist to the Man-In-The-Middle attack. In step four, I take the device UUID as the identifier instead of the public key is because there are various kinds of DSA, I have to united the ID format for all the users and all the devices.

\subsubsection{Login Phase}

\begin{enumerate}
\item User send a login request to server.
\item Server return a nonce ([server info || random bits]) back to client.
\item The NFC reader start to scan cards as soon as it receive the nonce.
\item User execute the card emulation application on mobile device and enter the PIN code. If the PIN code correct, mobile device enable the HCE mode.
\item Reader application send nonce to mobile device.
\item Mobile device retrieve the server info from nonce, show it on the screen and ask for user's confirmation.
\item Mobile device signed the nonce with corresponding private key.
\item Mobile device pass its UUID and signed-nonce to reader.
\item Client pass these parameters to server.
\end{enumerate}

Note that users have to swipe their device to the NFC reader in 30 seconds right after the reader start scanning. Otherwise, it will send a timeout message to browser. In step nine, verification server is able to ask user to provide some other credentials, but we will not discuss about that because it is not defined in my scheme.

\subsubsection{Verification Phase}

\begin{enumerate}
\item Server retrieve the corresponding public key according to the UUID.
\item Verify the signature with the public key.
\item Return the verification result back to cient.
\end{enumerate}

\section{Scenario}

\subsection{Future Website}
\label{sec:future-website}

\subsection{Existing Website}

For existing website, it is difficult for them to integrate this scheme with their origin users. Therefore, they can adopt the OpenID protocol to help. Build a verificaiton server with this new scheme and use it to be the \emph{Relying Party} in OpenID protocol.

Take Bitcucket.org as an example, I modified the verificaiton server in section~\ref{sec:future-website} to be an \emph{openid-provider}, that is, to support OpenID feature. When a user need to login to Bitbucket, he have to switch to openid-login mode, enter the server url. Then the authentication process is as same as I described above. 

\end{CJK}