\begin{CJK}{Bg5}{bsmi}

%---------------------------------------------
%	Chapter Preliminaries
%---------------------------------------------

\chapter{Preliminaries}

\section{Digital Signature Algorithm}

A digital signature is a mathematical scheme for demonstrating the authenticity of a digital message or document. 

A digital signature scheme typically consists of three algorithms:

1. A key generation algorithm that selects a private key uniformly at random from a set of possible private keys. The algorithm outputs the private key and a corresponding public key.
2. A signing algorithm that, given a message and a private key, produces a signature.
3. A signature verifying algorithm that, given a message, public key and a signature, either accepts or rejects the message's claim to authenticity.

Digital signatures can be used to authenticate the source of messages. When ownership of a digital signature secret key is bound to a specific user, a valid signature shows that the message was sent by that user. I take the advantage of this authentication property to replace password. In password-based authentication scheme, the ownership of password is bound to a specific user. 

\section{Android HCE Feature}

Many Android devices which offer NFC functionality also support card emulation feature. In most cases, this feature is achieved by a seperate chip, called \emph{secure element}. The Android system is only provide an interface. Therefore, no Android application can involved in the transaction between secure element and the reader. After the transaction complete, an applications can query the secure element directly to get transaction status and notify users. This is why Tte original NFC card emulaiton functionality also called hardware card emulation.

Because this mechanism needs an extra hardware in devices, most of Android application developers cannot take advantages of card emulation feature. To solve this, Android 4.4 provides an additional method of card emulation, which is not invoved with secure element, called \emph{host-base card emulation}. This method allows Android application can communicate with NFC reader directly, not 

\section{Existing Solutions}

Since researchers have studied authentication system for years, there are various of solution now. 

\subsection{Token-based scheme}

\subsubsection{SecurID}

SecurID, now known as RSA SecurID, is a mechanism developed by RSA (the Security Division of EMC) for performing two-factor authentication for a user to a network resource. The following paragragh will describe how it works simply.

Each device stored a defferent secret \emph{seed}, and the back-end server also know this seed. In every 60 seconds, SecurID will generate an 6-digit authentication code according to its seed, and display on the screen. If a user authenticating to a network service, he have to enter both a personal identificaiton number and the 6-digit \emph{code at that moment}. The server, which also has a real-time clock and a database of valid tokens with the associated seed records, authenticates a user by computing what number the token is supposed to be showing at that moment in time and checking this against what the user entered.

However, in March 2011 attackers compromised RSA's back-end database of seeds, which allowed them to predict the authentication codes generated by any token at any time. This attack forces RSA Security to replace almost every one of the 40 million SecurID tokens in use.

\subsubsection{YubiKey}

\subsection{Mobile-device-based scheme}

\subsubsection{Google 2-step}

\subsection{Smart-card-based scheme}

\subsubsection{Swarn's Smart-card-based Scheme}

\end{CJK}